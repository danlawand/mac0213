\documentclass{article}
\usepackage[utf8]{inputenc}
\usepackage[brazil]{babel}
\usepackage{url}
\usepackage{tabularx}
\usepackage{palatino}
\usepackage{hyperref}
\usepackage{pdfpages}
\usepackage[margin=2.5cm]{geometry}

\title{\textbf{MAC0213 - Atividade Curricular em Comunidade \\ Projeto: Inclusão digital de jovens em situação de vulnerabilidade social}}
\author{
    \textbf{Aluno:} Daniel Angelo Esteves Lawand  \\
    \textbf{Número USP:} 10297693
    }
\date{}

\begin{document}
\maketitle

\section{Introdução}

A Juventude Masculina do Movimento de Schoenstatt (JUMAS) realiza atividades de formação humana e social para jovens em diferentes cidades pelo Brasil. Na capital paulista, existe a unidade da JUMAS que opera no bairro do Jaraguá, uma região periférica da cidade de São Paulo. Nesse bairro, a JUMAS trabalha com jovens que vivem em situação de vulnerabilidade social, ou seja, muitos deles vivem em condições precárias de moradia e saneamento, com poucos meios de subsistência, com a ausência de um ambiente familiar e social equilibrado para o desenvolvimento humano e com pouco ou nenhum acesso à internet.

Em meio à pandemia do COVID-19, a vida desses jovens foi impactada de inúmeras maneiras, uma delas foi a exclusão social e digital que eles sofreram em resultado do isolamento social e da pobre conexão à internet. A JUMAS trabalha, em grande parte, com a inclusão social desses jovens, contudo sofre com falta de meios e de processos que permitem a inclusão digital do jovem do bairro do Jaraguá. Nesse sentido, esse projeto visa criar materiais que servirão como parte de mecanismos e processos que ajudarão à JUMAS realizar a inclusão digital desses jovens.

\section{Objetivos}

Ao aderir a esse projeto, o aluno investigará quais são as tecnologias mais necessárias para realizar a inclusão digital do público alvo, e, após ter definido essas tecnologias, o aluno desenvolverá tutoriais escritos e/ou em vídeos do como usar essas tecnologias. Tais tutoriais serão sobre o uso de tecnologias como pacote \textit{office}, algumas linguagens de programação, ferramentas de e-mail, \textit{setup} do sistema operacional e entre outras tecnologias. 

\section{Tarefas}

As atividades terão 105 horas de duração e serão divididas em 8 tarefas, da seguinte maneira:

\begin{center}
  \begin{tabular}{ | l | l | p{5cm} | l | }
    \hline
    \# & Tópico & Tarefa & Duração prevista \\ \hline
    1 & Apresentação & Elaboração do cronograma do projeto, juntamente ao supervisor & 6 horas \\ \hline
    2 & Identificação & Elaborar formulário de pesquisa para identificar as principais tecnologias & 10 horas \\ \hline
    3 & Identificação & Identificar as principais tecnologias, com pesquisa de campo e análise das respostas do formulário de pesquisa & 10 horas \\ \hline
    4 & Identificação & Selecionar as 4 principais tecnologias a serem usadas no projeto & 2 horas \\ \hline
    5 & Estudo & Estudo das 4 tecnologias selecionadas  & 16 horas \\ \hline
    6 & Resolução & Elaboração e entrega dos tutoriais das 4 tecnologias selecionadas \textbf{até 25 de junho de 2023} & 48 horas  \\ \hline
    7 & Apresentação & Elaboração de materiais de apresentação (por exemplo, \textit{PowerPoint}) para a explicação dos tutoriais aos voluntários da JUMAS & 5 horas  \\ \hline
    8 & Apresentação & Apresentação dos tutoriais aos voluntários da JUMAS \textbf{até 25 de junho de 2023} & 8 horas  \\ \hline
 \end{tabular}
\end{center}

De maneira geral, para cada um dos quatro tutoriais serão quatro horas de estudo e doze horas de preparação do material. A criação do material de apresentação deverá tomar uma hora e meia para cada tutorial. E será uma hora de apresentação de cada tutorial e mais uma hora para tirar dúvidas de cada tutorial.


\section{Cronograma mês a mês}

\subsection{Março}
\begin{center}
  \begin{tabular}{ | l | l | p{5cm} | l | }
    \hline
    \# & Tarefa & Duração prevista \\ \hline
    1 & Realização da Tarefa número 1 & 6 horas \\ \hline
    TOTAL && 6 horas \\ \hline
 \end{tabular}
\end{center}


\subsection{Abril}
\begin{center}
  \begin{tabular}{ | l | l | p{5cm} | l | }
    \hline
    \# & Tarefa & Duração prevista \\ \hline
    1 & Realização da tarefa número 2 & 10 horas \\ \hline
    2 & Realização da tarefa número 3 & 10 horas \\ \hline
    3 & Realização da tarefa número 4 & 2 horas \\ \hline
    4 & Estudo da tecnologia 1 & 4 horas \\ \hline
    5 & Início do tutorial da tecnologia 1 & 6 horas \\ \hline
    TOTAL && 32 horas \\ \hline
 \end{tabular}
\end{center}

\subsection{Maio}
\begin{center}
  \begin{tabular}{ | l | l | p{5cm} | l | }
    \hline
    \# & Tarefa & Duração prevista \\ \hline
    1 & Conclusão do tutorial da tecnologia 1 & 6 horas \\ \hline
    2 & Estudo da tecnologia 2 & 4 horas \\ \hline
    3 & Realização do tutorial da tecnologia 2 & 12 horas \\ \hline
    4 & Estudo da tecnologia 3 & 4 horas \\ \hline
    5 & Realização do tutorial da tecnologia 3 & 12horas \\ \hline
    TOTAL && 38 horas \\ \hline
 \end{tabular}
\end{center}


\subsection{Junho}
\begin{center}
  \begin{tabular}{ | l | l | p{5cm} | l | }
    \hline
    \# & Tarefa & Duração prevista \\ \hline
    1 & Estudo da tecnologia 4 & 4 horas \\ \hline
    2 & Realização do tutorial da tecnologia 4 & 12 horas \\ \hline
    4 & Realização da Tarefa número 7 & 5 horas \\ \hline
    5 & Realização da Tarefa número 8 & 8 horas \\ \hline
    TOTAL && 29 horas \\ \hline
 \end{tabular}
\end{center}


\section{Progresso}

O projeto será acompanhado e monitorado com base no cronograma previamente alinhado entre o aluno e o supervisor. As horas dedicadas ao projeto serão registradas em um \textit{log} que deverá ser assinado pelo supervisor e anexado ao relatório final.

Ao final do semestre, como requisito para a conclusão da disciplina, o aluno deverá produzir um relatório final sobre as atividades desenvolvidas.

O progresso atual pode ser conferido nesta página: \url{https://github.com/danlawand/mac0213}.

\section{Supervisor}

O responsável por supervisionar as atividades do aluno  será \textbf{Padre Gabriel Felipe Oberle}, assessor da JUMAS da unidade do bairro do Jaraguá, cujo e-mail é \texttt{gabriel.oberle@padresdeschoenstatt.org}.

\section{Carta de Concordância}

A carta de concordância do supervisor que concorda com a realização das atividades está anexada na próxima página.

\include{declaracao.jpeg}

\end{document}
