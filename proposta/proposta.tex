\documentclass{article}
\usepackage[utf8]{inputenc}
\usepackage[brazil]{babel}
\usepackage{url}
\usepackage{tabularx}
\usepackage{palatino}
\usepackage{hyperref}
\usepackage{pdfpages}
\usepackage[margin=2.5cm]{geometry}

\title{\textbf{MAC0213 - Atividade Curricular em Comunidade \\ Projeto: Inclusão digital de jovens em situação de vulnerabilidade social}}
\author{
    \textbf{Aluno:} Daniel Angelo Esteves Lawand  \\
    \textbf{Número USP:} 10297693
    }
\date{}

\begin{document}
\maketitle

\section{Introdução}

A Juventude Masculina do Movimento Apostólico de Schoenstatt (JUMAS) realiza atividades de formação humana e social para jovens em diferentes cidades pelo Brasil. Na capital paulista, existe a unidade do JUMAS que opera no bairro do Jaraguá, uma região periférica da cidade de São Paulo. Nesse bairro, o JUMAS trabalha com jovens que vivem em situação de vulnerabilidade social, ou seja, muitos deles vivem em condições precárias de moradia e saneamento, com poucos meios de subsistência, com a ausência de um ambiente familiar e social equilibrado para o desenvolvimento humano e com pouco ou nenhum acesso à internet.

Em meio à pandemia do COVID-19, a vida desses jovens foi impactada de inúmeras maneiras, uma delas foi a exclusão social e digital que eles sofreram em resultado do isolamento social e da pobre conexão à internet. O JUMAS trabalha, em grande parte, com a inclusão social desses jovens, contudo sofre com falta de meios e de processos que permitem a inclusão digital do jovem do bairro do Jaraguá. Nesse sentido, esse projeto visa criar materiais que servirão como parte de mecanismos e processos que ajudarão ao JUMAS realizar a inclusão digital desses jovens.

\section{Objetivos}

Ao aderir a esse projeto, o aluno deverá identificar frentes de trabalho cinco temáticas de análises de dados. Após o desenho do projeto, o aluno desenvolverá, com apoio de áreas internas relacionadas na SME, as análises quantitativas, tais como: 

Além disso, o aluno deverá desenvolver análises extremamente ricas, e que envolvam, além do código e \textit{deliverables} associados, um relatório analítico. 

Serão cinco temáticas de análises de dados alinhadas ao Banco de Desafios do Programa de Cooperação em Pesquisa.


\section{Tarefas}

As atividades terão 100 horas de duração, conforme a divisão a seguir:

\begin{center}
  \begin{tabular}{ | l | l | p{5cm} | l | }
    \hline
    \# & Tópico & Tarefa & Duração prevista \\ \hline
    1 & Apresentação & Desenhar, junto com o supervisor, cronograma do Projeto  & 3 horas \\ \hline
    2 & Identificação & Identificar cinco possíveis análises a partir dos microdados e outras bases já publicadas & 15 horas \\ \hline
    3 & Identificação & Identificar relações com o Banco de Desafios do Programa de Cooperação em Pesquisa & 5 horas \\ \hline
    4 & Participação & Realizar reuniões com áreas internas da SME para alinhamento e eventuais dúvidas & 7 horas \\ \hline
    5 & Resolução & Entregar 3 análises e relatórios técnicos \textbf{até novembro de 2018} & 40 horas  \\ \hline
    6 & Resolução & Entregar 2 análises e relatórios técnicos \textbf{até dezembro de2018} & 30 horas \\ \hline
 \end{tabular}
\end{center}

De maneira geral, para cada uma das cinco análises, as atividades se distribuem em duas horas de reuniões com a Secretaria, duas horas de planejamento e desenho da análise e dezesseis horas de realização da análise, distribuídas entre limpeza, seleção e organização dos dados, escolha do modelo, desenvolvimento do modelo, ajustes finais do modelo, análise dos resultados e escrita do relatório técnico. 

\section{Progresso}

O projeto será acompanhando e monitorado com base no cronograma previamente alinhado entre o aluno e o supervisor. As horas dedicadas ao projeto serão registradas em um \textit{log} que deverá ser assinado pelo supervisor e anexado ao relatório final.

Ao final do semestre, como requisito para a conclusão da disciplina, o aluno deverá produzir um relatório final sobre as atividades desenvolvidas.

O progresso atual pode ser conferido nesta página: \url{https://github.com/danlawand/mac0213}.

\section{Supervisor}

O responsável por supervisionar as atividades do aluno  será \textbf{Gabriel Oberle}, \textbf{assessor da Juventude Masculina de Schoenstatt do bairro do Jaraguá}, cujo e-mail é \texttt{gabriel.oberle@padresdeschoenstatt.org}.

\section{Carta de Concordância}

A carta de concordância do supervisor que concorda com a realização das atividades está anexada na próxima página.

\includepdf{declaracaoanalisedados.pdf}

\end{document}
